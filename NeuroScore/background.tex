\documentclass[12pt]{article}
\usepackage[margin = 2.4cm]{geometry} % For margins of 3cm
\usepackage{graphicx}
\usepackage{float} % For H float position
\usepackage{gensymb} % For some symbols
\usepackage{amsfonts, amssymb, amsmath} % All three for maths symbols
\usepackage[export]{adjustbox} % For figure frames
\setlength{\parskip}{6pt} % To make nice looking paragraph spacing
\usepackage[export]{adjustbox} % For figure frames
\usepackage{rotating}
\usepackage[section]{placeins}
\usepackage{setspace} % For double spacing
\usepackage{pdfpages} % Allows including PDFs
\usepackage[sort&compress]{natbib} % bibliographies
\doublespacing


\title{Neurodevelopmental gene dysregulation score}


\begin{document}

\maketitle

\paragraph{Aim}
~\\ Use genes that are casual in monogeneic neurodevelopmental disorders and to see if they as a group are dysregulated in the brain of ASD patients

\section{Paper Summary}

\begin{itemize}
    \item In summary, single-nucleus RNA seq of the cortical tissue of patients with autism was used to identify autism associated changes in specific cell types 
    \item Samples included prefrontal cortext and anterior cingulate cortext 
    \item 15 ASD patients and 16 controls 
    \item generated 104,559 single nuclei gene expression profiles (52,556 from controls and 52,003 from ASD patients)
    \item Identified 17 cell types 
    \item neurons expressed more genes and transcripts than glia 
    \item Nuclear profiles from ASD and control subjects for each cell type using a inear mixed model 
    \item 692 differentially expression events were found which included 513 unqiue differentially expressed genes 
    \item 79\% of differentially expressed genes were coming from a single cell type
    \item Intersection of DEGs with the SFARI database was made and found a 13\% overlap (75 genes)
    \item SFARI genes were most overreepresented with L2/3 and L4 excitatory neurons 
    \item 
    
\end{itemize}

\section{Method}
\begin{itemize}
    \item Obtain the single-cell data from Velmeshev et al 
    \item Make a list of neurodevelopmental disease genes with subcategories 
    \item For each gene: 
    \begin{itemize}
        \item Calculate the z-score of its expression in a given individual and cell type relative to all cells from controls in the same cell-type 
        \item Box-plot the z-scores and calculate a wilcoxon's rank sums test comparing NDDs and non-NDD genes for each individual
    \end{itemize}
\end{itemize}
 

\section{Tasks to do}
\begin{itemize}
    \item Download raw data
    \item make a list of NDD and non-NDD genes and divide into subcategories 
    \item plot mean gene expression to see what the rate of expression is like 
    \item check variance of expression
    \item in which cell types are NDD cells mainly expressed 
    \item variance of expression in NDD vs non-NDD genes 
    \item average out expression and calculate z-score for each gene in each cell type for one ASD individual vs control 
    \item make boxplot of expression
    \item calculate wilcoxon ranks sums test between NDD and non-NDD genes 
\end{itemize}
\end{document}