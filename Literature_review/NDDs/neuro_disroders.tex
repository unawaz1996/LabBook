\documentclass[12pt]{article}
\usepackage[margin = 2.4cm]{geometry} % For margins of 3cm
\usepackage{graphicx}
\usepackage{float} % For H float position
\usepackage{gensymb} % For some symbols
\usepackage{amsfonts, amssymb, amsmath} % All three for maths symbols
\usepackage[export]{adjustbox} % For figure frames
\setlength{\parskip}{6pt} % To make nice looking paragraph spacing
\usepackage[export]{adjustbox} % For figure frames
\usepackage{rotating}
\usepackage[section]{placeins}
\usepackage{setspace} % For double spacing
\usepackage{pdfpages} % Allows including PDFs
\usepackage[sort&compress]{natbib} % bibliographies
\doublespacing




\begin{document}

\section{Neurodevelopmental disorders}
Neurodevelopment is the biological process resulting in the development and maturation of the nervous system.
In humans, the process starts at the third week of embryonic growth with the formation of the neural tube.
 From the ninth week onward, the brain orderly maturates and acquires its typical structure, under a tightly orchestrated chain of events that includes abundant cell proliferation, migration, and differentiation [1, 4, 5]. 
 Any disruption to such orderly and complex chain of events may lead to dysfunctional brain development, and consequently to a neurodevelopmental phenotype.
  Under the designation neurodevelopmental disorders (NDDs) falls a group of complex and heterogeneous disorders showing symptoms associated to abnormal brain development that may give rise to impaired cognition, communication, adaptive behavior, and psychomotor skills [6,7,8]. 
NDDs include, for example, autism spectrum disorder, intellectual disability, attention deficit hyperactivity disorder [7, 9, 10]. 
The prevalence of these disorders constitutes a serious health problem in modern days. 
Previous reviews in distinct populations indicated a median global estimate of 62/10,000 for autism [11], 10.37/1000 for intellectual disability [12]
\subsection{Autism spectrum disorders}
Autism spectrum disorder (ASD is a group of NDDs charactertized by a high degree of clinical and genetic heterogeneity,
Among the human brain regions implicated in the pathophysiology of ASD, the prefrontral cortex has always been focused on due to its role in the cognitive, decision-making and social behavior.


\end{document}