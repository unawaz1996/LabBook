\documentclass[12pt]{article}
\usepackage[margin = 2.4cm]{geometry} % For margins of 3cm
\usepackage{graphicx}
\usepackage{float} % For H float position
\usepackage{gensymb} % For some symbols
\usepackage{amsfonts, amssymb, amsmath} % All three for maths symbols
\usepackage[export]{adjustbox} % For figure frames
\setlength{\parskip}{6pt} % To make nice looking paragraph spacing
\usepackage[export]{adjustbox} % For figure frames
\usepackage{rotating}
\usepackage[section]{placeins}
\usepackage{setspace} % For double spacing
\usepackage{pdfpages} % Allows including PDFs
\usepackage[sort&compress]{natbib} % bibliographies
\doublespacing




\begin{document}

\section{Neurodevelopmental disorders}
The development of the human brain is a complex and a tightly regulated process during which changes occur at both
anatomical and functional levels (Silbereis et al., 2016). The processes of brain development are highly
dependent on the appropriate expression of RNA and proteins. Mutations altering the expression of genes involved in brain development can cause or contribute to neurodevelopmental disorders (NDDs)
(Tebbenkamp et al ., 2014).


10% to 15% of population with prevalence rates increasing worldwide.
\subsection{Autism spectrum disorders}
Autism spectrum disorder (ASD is a group of NDDs charactertized by a high degree of clinical and genetic heterogeneity,
Among the human brain regions implicated in the pathophysiology of ASD, the prefrontral cortex has always been focused on due to its role in the cognitive, decision-making and social behavior.


\end{document}