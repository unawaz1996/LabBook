\documentclass[12pt]{article}
\usepackage[margin = 2.4cm]{geometry} % For margins of 3cm
\usepackage{graphicx}
\usepackage{float} % For H float position
\usepackage{gensymb} % For some symbols
\usepackage{amsfonts, amssymb, amsmath} % All three for maths symbols
\usepackage[export]{adjustbox} % For figure frames
\setlength{\parskip}{6pt} % To make nice looking paragraph spacing
\usepackage[export]{adjustbox} % For figure frames
\usepackage{rotating}
\usepackage[section]{placeins}
\usepackage{setspace} % For double spacing
\usepackage{titling}
\usepackage{lscape}
\usepackage{courier}
\usepackage[sort&compress]{natbib} % bibliographies
\doublespacing


\begin{document}
    \section{The use of transcriptomics to study the brain}
    Although the genetic etiology of NDDs is far from being completely known, significant advances have been made in the last years in understanding specific biological pathways underlying the molecular mechanisms of these illnesses. 
    \begin{itemize}
        \item What is single cell and how has it been used 
        \item in this section I will describe several studies that utilised this single cell rna-seq approach 
    \end{itemize}
The advances in next generertion sequening technologies in recent years have provided many valuable insights into complex biological systems. NGS

\subsection{Single-cell RNA-seq}
Since the first paper showing the feasibility of characterizing expression profiles of individual cells using NGS technologies was demonstrated by Tang et al, single-cell RNA sequencing has been widely used to dissect the cellular heterogeneity within a population of cells. Single-cell sequencing technologies refer to the sequencing of the genome, transcriptome or the epigenome at a single-cell level and allows the discovery of obscured populations of cells. 

The great potential for this technology has generated an interest in obtaining high resolution of single cells which can help identify rare populations that cannot be detected from bulk analysis. In this section, we will discuss the advances made in single-cell technologies in the recent years, and how these advances have helped our understaning in various aspects of neurodevelopmental disorders. 




\subsubsection{Single-cell sequencing platforms}
Normally, next-generation sequencers require the input DNA to be at a nanogram level which is orders of 




\subsubsection{Application of scRNA-seq in the brain}
The  brain contains highly complex neural cell types and subtypes which are organized in specialised regions. Traditionally, neural cells were identified by morphology, excitability, connectivity and the location of the cell. However, with recent advamces, scRNA-seq approaches have become a common tool to assess and investigate the brains complexity and to identify new subpopulations. From 2015 onwards over 80 papers have reported detailed characterization of brain cell types in different brain regions, and at developmental stages or disease status using scRNA-seq. In addition to the increasing number of publications, we have also observed an exponentially increasing number of sequenced cells per study in the last 5 years. The technology is not only inspiring more studies in recent years, but also exponentially scaling up the number of single cells profiled in each study, which has empowered the construction of a comprehensive landscape of the cell types in the brain.


\begin{landscape}

\begin{table}[!htp]
	\centering
	\footnotesize
	\caption{Expression studies that utilised single-cell profiling technologies (Not ordered by any particular way)}
	\begin{tabular}{|l|l|l|l|l|l|l|l|}
		\hline
		\textbf{Authors} &\textbf{Year}  & \textbf{Tissues} & \textbf{Method} & \textbf{No. of Cells/Nuclei}  & \textbf{Age} &  \textbf{No. of individuals} & \textbf{Data availability}  \\ \hline
		Velmeshev et al & 2019  &  PFC and ACC  & snRNA-seq  & 104,559 & 4-22 pm &15 ASD; 16 controls; total 31&  \\ \hline
		Late et al& 2016  & Brodmann's areas  & snRNA-seq & 4488 &51 pm & 1 healthy individual &  \\ \hline
		Nowakozski et al & 2017 &V1, PFC & scRNA-seq     & &&& \\ \hline
	\end{tabular}
\end{table}
\end{landscape}


\paragraph{single-cell nuclei technique}

~\\ Points to discuss: The first study that used it and why it was neccesary \\

The more recent study and hwo it used it in NDDs


\end{document}
