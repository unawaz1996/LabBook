\documentclass[12pt]{article}
\usepackage[margin = 2.4cm]{geometry} % For margins of 3cm
\usepackage{graphicx}
\usepackage{float} % For H float position
\usepackage{gensymb} % For some symbols
\usepackage{amsfonts, amssymb, amsmath} % All three for maths symbols
\usepackage[export]{adjustbox} % For figure frames
\setlength{\parskip}{6pt} % To make nice looking paragraph spacing
\usepackage[export]{adjustbox} % For figure frames
\usepackage{rotating}
\usepackage[section]{placeins}
\usepackage{setspace} % For double spacing
\usepackage{pdfpages} % Allows including PDFs
\usepackage[sort&compress]{natbib} % bibliographies
\doublespacing


\begin{document}
    \section{Use of expression studies in deciphering NDDs and the brain in general}
    Although the genetic etiology of NDDs is far from being completely known, significant advances have been made in the last years in understanding specific biological pathways underlying the molecular mechanisms of these illnesses. 
    \begin{itemize}
        \item What is single cell and how has it been used 
        \item in this section I will describe several studies that utilised this single cell rna-seq approach 
    \end{itemize}

Since the first paper showing the deasibility of characterizing the transcriptomes of individual cells was published by 
\end{document}