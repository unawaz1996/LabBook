\documentclass[12pt]{article}
\usepackage[margin = 2.4cm]{geometry} % For margins of 3cm
\usepackage{graphicx}
\usepackage{float} % For H float position
\usepackage{gensymb} % For some symbols
\usepackage{amsfonts, amssymb, amsmath} % All three for maths symbols
\usepackage[export]{adjustbox} % For figure frames
\setlength{\parskip}{6pt} % To make nice looking paragraph spacing
\usepackage[export]{adjustbox} % For figure frames
\usepackage{rotating}
\usepackage[section]{placeins}
\usepackage{setspace} % For double spacing
\usepackage{pdfpages} % Allows including PDFs
\usepackage[sort&compress]{natbib} % bibliographies
\doublespacing


\begin{document}

\section{Introduction}
\section{Neurodevelopmental disoders}
\section{Co-morbidity of NDDs}
\section{Expression studies}

\section{Thesis aims}

\begin{itemize}
    \item To comprehensively characterize the expression properties of ID and CP enes in the normal human brain
    \begin{itemize}
        \item To determine whether ID and CP genes are expressed in a cell-type speciic mannner using single-cell RNA-seq data
        \item To characterize the developmental trajectory of gene expression for ID and CP genes and assess their expression during cellular maturation in brain organiods and assess their expression across multiple fetail developmental periods 
        \item  To characterize the spatial and temporal properties of ID and CP gene expression in the adult brain, by assessing age-dependent changes  brain-region and cortical layer-specificity 
        
    \end{itemize}

    \item To determine whether convergent gene expression changes are observed in ID and CP in patient-derived cells with heterogenous mutations 

\end{itemize}

\end{document}
