\documentclass[12pt]{article}
\usepackage[margin = 2.4cm]{geometry} % For margins of 3cm
\usepackage{graphicx}
\usepackage{float} % For H float position
\usepackage{gensymb} % For some symbols
\usepackage{amsfonts, amssymb, amsmath} % All three for maths symbols
\usepackage[export]{adjustbox} % For figure frames
\setlength{\parskip}{6pt} % To make nice looking paragraph spacing
\usepackage[export]{adjustbox} % For figure frames
\usepackage{rotating}
\usepackage[section]{placeins}
\usepackage{setspace} % For double spacing
\usepackage{pdfpages} % Allows including PDFs
\usepackage[sort&compress]{natbib} % bibliographies
\doublespacing

\title{Spatiotempotal properties of NDD genes during brain development for biomarker discovery}
\author{Urwah Nawaz}


\begin{document}

\begin{titlepage}
    \maketitle
\end{titlepage}


\section{Aims of the project}

The aims of this project are: 

1) To define and explore cell type-specific expression patterns of NDD genes via a meta-analysis of brain cell type RNA-expression data 

2) To characterize the developmental trajectory of gene expression for ID and CP genes and assess their expression during cellular maturation in brain organiods and assess their expression across multiple fetal developmental periods

what to review

\begin{itemize}
    \item human brain single cell rna-seq studies 
    \item reread papers used for PhD proposal 
    \item number and types of samples 
    \item method used for sequencing 
    \item data availablity 
    \item main conclusions of the studies
    \item co-expression networks and their utility in spatiotemporal studes 
    \item this kind of analysis in other areas (particularly cancer)
    \item Use of single-cell analysis vs bulk rna-seq 
\end{itemize}

section structure: 
\begin{itemize}
    \item Introduction
    \item Neurodevelopmental disorders 
    \item RNA-sequencing studies and its use in brain 
    \item Utility of single cell in this area  
    \item co-expression analyses amd their application in NDDs
    \item Gene lists for NDD 
    \item Lack of gene networks in NDDs
    \item 
\end{itemize}

\begin{itemize}
    \item 
\end{itemize}

\section{Introduction}
Human brain development is a complex and a tightly regulated process duroing which changes occur at both anatomical and functional levels.
The processes of brain development are highly dependent on the appropriate expression of RNA and proteins, 
Mutatations that result in altered expression or function of these gene products can cause or contribute to neurodevelopmental disorders (NDDs). 
\begin{itemize}
    
\end{itemize}


\section{Neurodevelopmental disorders}

Neurodevelopmental disorders (NDDs) such as autism spectrum disorder, intellectual disability and epilipsy are characterized by abnormal brain development.
High co-occurance of NDDs indicate a shared, underlying biological mechanism. 
The genetic heterogeneity and overlap observed in NDDs make it difficult to identify the genetic causes of specific clinical symptoms
\section{Co-mobidity of NDDs}
\section{Expression studies}
\begin{itemize}

 \item what are the datasets available 
\item where am i going to get the genelists from 
\item what gene ontologies am I getting with my genes 
\item current co-expression networks that are available
\item lack of ID gene networks 
\item benchmark potentially?
\item Issues with bulk rna? 

\end{itemize}
\section{Thesis aims}

\begin{itemize}
    \item To comprehensively characterize the expression properties of ID and CP enes in the normal human brain
    \begin{itemize}
        \item To determine whether ID and CP genes are expressed in a cell-type speciic mannner using single-cell RNA-seq data
        \item To characterize the developmental trajectory of gene expression for ID and CP genes and assess their expression during cellular maturation in brain organiods and assess their expression across multiple fetail developmental periods 
        \item  To characterize the spatial and temporal properties of ID and CP gene expression in the adult brain, by assessing age-dependent changes  brain-region and cortical layer-specificity 
        
    \end{itemize}

    \item To determine whether convergent gene expression changes are observed in ID and CP in patient-derived cells with heterogenous mutations 

\end{itemize}

\section{Characterisation of brain diversity using single-cell}
\subsection{Studies that have characterized the single-cell transcriptome in the brain}

\begin{table}[]
    \begin{tabular}{llllllll}
     Year & Cells reported  & Method  & Technique  & Species  &  Cell isolation & Brain region & Developmental stages\\
     2014 &  &  &  & & & & \\
     &  &  &  & &&& \\
     &  &  &  & &&&
    \end{tabular}
    \end{table}


\end{document}
